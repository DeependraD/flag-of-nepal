\documentclass[tikz,border=10pt]{standalone}

% \usetikzlibrary{calc,positioning,intersections}
% 
% \begin{document}
% \begin{tikzpicture}
% 
% % Define the coordinates for your path
% \coordinate (A) at (0, 0);
% \coordinate (B) at (2, 1);
% \coordinate (E) at (3, 1);
% \coordinate (G) at (2, 2);
% \coordinate (C) at (0, 2);
% 
% % Stroke draws both inside and outside of path
% \path[draw,thin,save path=\cshape] (A) -- (B) -- (E) -- (G) -- (C) -- cycle;
% \path[fill=red!50,use path=\cshape];
% \fill[white] (0.2, 0.3) -- (0.8, 0.6) -- (0.4, 1) -- cycle;
% \def\lenTN{3.5} % assume ! (it is infact computed)
% \path[draw=black,thin,use path=\cshape]; % display purpose only
% \path[draw,line width=\lenTN,blue!50,opacity=0.6,use path=\cshape];

\usetikzlibrary{calc,positioning,intersections,decorations}
\pgfdeclaredecoration{stroke outside path}{initial}{%
  \state{initial}[width=\pgfdecoratedinputsegmentlength/100,next state=iterate]
  {
    \pgfpathmoveto{\pgfqpoint{0pt}{.5\pgflinewidth}}
  }%
  \state{iterate}[width=\pgfdecoratedinputsegmentlength/100]
  {
    \pgfpathlineto{\pgfqpoint{.5\pgflinewidth}{.5\pgflinewidth}}
  }%
  \state{final}{%
    \pgfpathlineto{\pgfpointadd{\pgfqpoint{\pgflinewidth}{.5\pgflinewidth}}{\pgfpointdecoratedpathlast}}
  }%
}%


\begin{document}
\begin{tikzpicture}

% Define the coordinates for your path
\coordinate (A) at (0, 0);
\coordinate (B) at (2, 1);
\coordinate (E) at (3, 1);
\coordinate (G) at (2, 2);
\coordinate (C) at (0, 2);

% Stroke draws both inside and outside of path
\path[fill=red!50] (A) -- (B) -- (E) -- (G) -- (C) -- cycle;
\fill[white] (0.2, 0.3) -- (0.8, 0.6) -- (0.4, 1) -- cycle;
\def\lenTN{3.5} % assume ! (it is infact computed)
% \path[draw=black,thin,use path=\cshape]; % display purpose only
\draw [decoration=stroke outside path, decorate, line width = \lenTN] (A) -- (B) -- (E) -- (G) -- (C) -- cycle;
\draw [red](A) -- (B) -- (E) -- (G) -- (C) -- cycle;



\end{tikzpicture}
\end{document}
