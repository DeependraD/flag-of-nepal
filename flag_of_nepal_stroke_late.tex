% https://0xc.de/flags/nepal/
\documentclass[tikz,border=0.5cm]{standalone}
\usetikzlibrary{positioning,calc,shapes.geometric,intersections,through,decorations.pathmorphing,decorations.shapes}
\usetikzlibrary{ext.paths.arcto}
\usetikzlibrary{nfold, ext.misc}
\makeatletter
\tikzset{offset/.code=\tikz@addmode{\pgfgetpath\tikz@temp\pgfsetpath\pgfutil@empty\pgfoffsetpath\tikz@temp{#1}}}
\makeatother

\begin{document}

\begin{tikzpicture}[scale=2]

% Define colors
\definecolor{nepalred}{RGB}{206,17,38} % Crimson red
\definecolor{nepalblue}{RGB}{0,56,168} % Blue border

% Dimensions of the triangles
\def\base{3} % Base of the small triangle
\def\height{4} % Height of the small triangle

\coordinate (A) at (0,0);
\coordinate (B) at (\base,0);
\coordinate (C) at (0,\height);
\coordinate (D) at (0,\base);
\coordinate (E) at ({3-3/sqrt(2)},{3/sqrt(2)});
\coordinate (F) at (0,{3/sqrt(2)});
\coordinate (G) at (3,{3/sqrt(2)});
\coordinate (H) at (3/4,0);
\coordinate (I) at (3/4,{3 + 3/4/sqrt(2)}); % note do not use command slash in sqrt

% \fill[nepalred,save path=\lowertri] (A) -- (B) -- (D) -- cycle;
% \fill[nepalred,save path=\uppertri] (C) |- (E) -- (G) -- cycle;

% % Labelling nodes for convenience
% \node at (A) {A};
% \node at (B) {B};
% \node at (C) {C};
% \node at (D) {D};
% \node at (E) {E};
% \node at (F) {F};
% \node at (G) {G};
% \node at (H) {H};
% \node at (I) {I};

\coordinate (J) at ($ (C)!0.5!(F) $);
\coordinate (K) at ($ (C)!0.5!(G) $);

% \node at (J) {J};
% \node at (K) {K};

\path[name path=JK] (J) -- (K);

\path[draw=none,thin,blue!40,dashed,name path=JK] (J) -- (K);
\path[draw=none,thin,blue!40,dashed,name path=HI] (H) -- (I);

% % hide
% \path[name path=JG] (J) -- (G); % this path is useful only when using path to find intersection
% \path[name path=DB] (D) -- (B);

% Find and name intersection
\path[name intersections={of=JK and HI, by=L}];
\coordinate (M) at (intersection of J--G and H--I); % note H--I syntax is new
\coordinate (X) at (intersection of D--B and H--I); %

% % hide: Draw and mark the midpoint
% \draw[fill=white] (L) circle (0.5pt) node[very thin,below] {L};
% \draw[fill=white] (M) circle (0.5pt) node[very thin,below] {M};
% \draw[fill=white] (X) circle (0.5pt) node[very thin,below] {X};

\draw let \p1=($(M)-(X)$), \n1={veclen(\x1,\y1)} in coordinate (N) at ($(M)!\n1/sqrt(2)!(X)$);
% % hide
% \draw[fill=white] (N) circle (0.5pt) node[very thin,below] {N};

%% difficult to clip the shapes with nodes
% \node[name path=CimpM,draw,circle through=(X)] at (M) {}; % uses tikzlibrary "through"
% \node[name path=CimpL,draw,circle through=(N)] at (L) {};
% \node[name path=CimpN,draw,circle through=(M)] at (N) {};

\path[draw=none,name path=CimpM] let \p1 = ($(M) - (X)$), \n1={veclen(\x1,\y1)} in \pgfextra{\xdef\rMX{\n1}} (M) circle [radius=\rMX];
\path[draw=none,name path=CimpL] let \p1 = ($(L) - (N)$), \n1={veclen(\x1,\y1)} in \pgfextra{\xdef\rLN{\n1}} (L) circle [radius=\rLN];
\path[draw=none,name path=CimpN] let \p1 = ($(N) - (M)$), \n1={veclen(\x1,\y1)} in \pgfextra{\xdef\rNM{\n1}} (N) circle [radius=\rNM];

%% Complex filling tricks in TikZ: https://tex.stackexchange.com/questions/614043/how-to-draw-diagram-of-circles-and-curves-with-pattern-filled-intersections-usi 
% \begin{scope}
% \path[clip] let \p1 = ($(N) - (M)$) in (M) circle [radius={veclen(\x1,\y1)}];
% \path[fill=white] let \p1 = ($(L) - (N)$) in (L) circle [radius={veclen(\x1,\y1)}];
% \end{scope}

% % hide
% \draw (J) -- (L);
% \draw (N) |- (J);

\path [draw=none,name path=OM] ($(C)!(M)!(F)$) coordinate (O) -- (M);
% \node at (O) {O};
\path [draw=none,name path=ExtendM] (M) -- +({\base-1},0);
\path [draw=none,name intersections={of=OM and CimpM, by=P}];
\path [draw=none,name intersections={of=ExtendM and CimpM, by=Q}];

% % hide
% \draw[fill] (P) circle (0.5pt) node {P};
% \draw[fill] (Q) circle (0.5pt) node {Q};

\path [draw=none,name intersections={of=CimpN and CimpL, name=cints}]; % circle intersections small circle
% % hide
% \foreach \n in {1,2} {
%     \fill[black] (cints-\n) circle (0.5pt) node[font=\tiny,below] {cs\n};
% }

\path[draw=none,name path=c1--c2] (cints-1) -- (cints-2); 
\path[draw=none,name path=MN] (M) -- (N); 

\path[draw=none,name intersections={of=c1--c2 and MN, by=T}];

% % hide
% \draw[fill] (T) circle (0.5pt) node {T};

% % hide
% \draw (T) circle (0.5pt);

% a larger concentric circle

\path[name path=segTN] let \p1 = ($(T) - (N)$), \n1 = {veclen(\x1,\y1)} in \pgfextra{\xdef\lenTN{\n1}} (T) -- (N); % radius/border
\path[name path=segTM] let \p1 = ($(T) - (M)$), \n1 = {veclen(\x1,\y1)} in \pgfextra{\xdef\lenTM{\n1}} (T) -- (M);
\path[draw=none,name path=CimpTB] (T) circle (\lenTN+\lenTM);
\path[draw=none,name path=CimpTS] (T) circle (\lenTM);

\path [draw=none,name intersections={of=CimpTB and CimpL, name=cintb}]; % circle intersections big circle
% % hide
% \foreach \n in {1,2} {
%     \fill[black] (cintb-\n) circle (0.5pt) node[font=\tiny,below] {cb\n};
% }

% Circle intersecting circle
\path [draw=none,name intersections={of=CimpL and CimpM, name=CI}];

% % hide: Draw the intersections
% \foreach \n in {1,2} {
%     \fill[black] (CI-\n) circle (0.5pt) node[font=\tiny,below] {CI\n};
% }

% Moon
\draw[draw=none,fill=white,save path=\moon] (CI-1) arc to[radius=(\rNM)](CI-2)
  arc to[radius=(\rLN),clockwise](CI-1) ;

\def\teeth{16}
\def\innerR{\lenTM}
% Add the lengths
\pgfmathsetmacro{\tempLength}{\lenTM + \lenTN}
% Pass the result to \def
\def\outerR{\tempLength pt}
\pgfmathsetmacro\angle{360/\teeth}

% best way to draw a spiked circle (no open path): https://tex.stackexchange.com/a/731097/133229
\draw[draw=none,save path=\star] ($(T)+(0:\innerR)$) \foreach \i in {1,...,\teeth} { -- ($(T)+({(\i-0.5)*360/\teeth}:\outerR)$) -- ($(T)+(\i*360/\teeth:\innerR)$)} -- cycle;

\begin{scope}
\clip ($(T)+(\outerR,-\outerR)$) rectangle ($(T)+(-\outerR,\outerR)$) ($(T)+(-\outerR,-\outerR)$) -- ($(T)+(-\outerR,0)$) -- (CI-1) arc to[radius=0.5*(\rNM+\rLN)] (CI-2) -- ($(T) + (\outerR,0)$) -- ($(T)+(\outerR,-\outerR)$) -- cycle; %replace \clip with \fill[red] to see area
\fill[white, use path=\star];
\end{scope}

% Sun
% \coordinate (U) at ($ (A)!0.5!(F) $);
\coordinate (W) at (3/4,{3/2/sqrt(2)});

\def\teeth{12}
\def\innerR{\rNM}
\def\outerR{\rLN}
\pgfmathsetmacro\angle{360/\teeth)}

% not a good way to draw a spiked circle (leaves open path) 
\fill[white] (W) circle [radius=\innerR];
\fill[white]
  \foreach \i in {1,2,...,\teeth} {%
     [rotate=(\i-1)*\angle]  ($(W) + (0:\innerR)$) -- ($(W) + (\angle*1/2:\outerR)$) -- ($(W) + (\angle:\innerR)$)
   };



\end{tikzpicture}

\end{document}
